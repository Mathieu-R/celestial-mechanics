\documentclass[11pt,twoside=semi,openright,numbers=noenddot]{article}

\def\ncode {LPHYS1351}
\def\ncourse {Projet différences finies}
\def\nlecturer {M. Crucifix}
\def\nyear {2021}
\def\ntitle {Mécanique Céleste}

\usepackage{amsmath}
\usepackage{amssymb}
\usepackage{amsthm}
\usepackage[T1]{fontenc}
\usepackage[french]{babel} % traduction
\usepackage[usenames,svgnames,dvipsnames]{xcolor}
\usepackage[hang, small, labelfont=bf, up, textfont=it]{caption} % custom caption under figures
%\usepackage{bookmark}
%\usepackage{booktabs} % beautiful tables
%\usepackage[thicklines]{cancel} % cancel terms in equation
%\usepackage{color} % colors in box (theorems,...)
%\usepackage{enumerate}
\usepackage{float}
\usepackage{framed} % box around text
\usepackage{fancyhdr} % header
%\usepackage{fancyvrb}
\usepackage{mathrsfs}
\usepackage{mathtools}
\usepackage[parfill]{parskip} % avoid bad alignment in paragraphs
\usepackage{pgfplots} % plot in tikz
\usepackage{braket} % braket and sets
\usepackage{derivative} % derivatives
\usepackage{siunitx}
\usepackage{silence} % silence useless warnings
\usepackage{tikz}
\usepackage{tkz-euclide}
\usepackage{tikz-3dplot}
\usepackage{units} % Non-stacked fractions and better unit spacing
\usepackage{xspace} % prints a trailing space in a smart way.

\usetikzlibrary{calc,patterns,angles,quotes}

\tikzset{circ/.style = {fill, circle, inner sep = 0, minimum size = 3}}
\tikzset{scirc/.style = {fill, circle, inner sep = 0, minimum size = 1.5}}
\tikzset{mstate/.style={circle, draw, blue, text=black, minimum width=0.7cm}}

% \titlespacing*{\section}{0pt}{5.5ex plus 1ex minus .2ex}{4.3ex plus .2ex}
% \titlespacing*{\subsection}{0pt}{5.5ex plus 1ex minus .2ex}{4.3ex plus .2ex}

\usepackage{enumitem}
\setlist{nosep,after=\vspace{\baselineskip}}

\pgfplotsset{compat=1.12}
\sisetup{locale = FR}

\usepackage[marginparwidth=2cm]{geometry}
\geometry{
	paper=a4paper,
	inner=2.0cm,
	outer=3.0cm,
	bindingoffset=.5cm,
	top=3.5cm,
	bottom=3.5cm
}

\usepackage{graphicx}
\setkeys{Gin}{width=\linewidth,totalheight=\textheight,keepaspectratio}
%\graphicspath{{figures/}}

% Default images settings
\setkeys{Gin}{width=\linewidth, totalheight=\textheight, keepaspectratio}

%---------------------------------------------------
% BUGGY PACKAGES THAT NEED TO BE LOADED AT THE END
%---------------------------------------------------

% colored hyperlink
% load hyperref at the end to avoid conflicts
\usepackage[colorlinks]{hyperref} % colored ref
% % cleverref must be loaded after hyperref
% \usepackage{cleveref} % create ref

\hypersetup{
  colorlinks=true,
  linkcolor=blue,
  filecolor=blue,
  citecolor=black,
  urlcolor=cyan
}

% conditions for equations
% https://tex.stackexchange.com/questions/95838/how-to-write-a-perfect-equation-parameters-description
\newenvironment{conditions}
  {\par\vspace{\abovedisplayskip}\noindent\begin{tabular}{>{$}l<{$} @{${}={}$} l}}
	{\end{tabular}\par\vspace{\belowdisplayskip}}

\newenvironment{conditions*}
  {\par\vspace{\abovedisplayskip}\noindent
   \tabularx{\columnwidth}{>{$}l<{$} @{${}={}$} >{\raggedright\arraybackslash}X}}
  {\endtabularx\par\vspace{\belowdisplayskip}}

%----------------
% FIX
%----------------

\setlength{\headheight}{14.5pt}

% increase vertical space for aligned equations
\setlength{\jot}{7pt}

% Filter warnings issued by package biblatex starting with "Patching footnotes failed"
\WarningFilter{biblatex}{Patching footnotes failed}

%----------------------
%	COMMANDS
%----------------------

% commands shortcuts
\newcommand{\mb}{\mathbb}
\newcommand{\R}{\mb{R}}
\newcommand{\Z}{\mb{Z}}
\newcommand{\N}{\mb{N}}
\newcommand{\C}{\mb{C}}
\newcommand{\A}{\mb{A}} % hypersphere area
\newcommand{\V}{\mb{V}} % hypersphere volume
\newcommand{\dS}{\cdot d\vb{S}}
\newcommand{\lag}{\mathcal{L}}
\newcommand{\ham}{\mathcal{H}}
\newcommand{\Mod}[1]{\ \mathrm{mod}\ #1}

\renewcommand{\vec}[1]{\mathbf{#1}} % bold vectors
\newcommand{\vd}[1]{\dot{\vec{#1}}}
\newcommand{\vdd}[1]{\ddot{\vec{#1}}}

% norm
\newcommand{\bignorm}[1]{\left\lVert#1\right\rVert}

% smaller overline (line above variable)
\newcommand{\overbar}[1]{\mkern 1.5mu\overline{\mkern-1.5mu#1\mkern-1.5mu}\mkern 1.5mu}

% prints an asterisk that takes up no horizontal space.
% useful in tabular environments.
\newcommand{\hangstar}{\makebox[0pt][l]{*}}

% Prints argument within hanging parentheses (i.e., parentheses that take
% up no horizontal space). Useful in tabular environments.
\newcommand{\hangp}[1]{\makebox[0pt][r]{(}#1\makebox[0pt][l]{)}}

% cancel terms with color
% \newcommand{\ccancel}[2]{\renewcommand{\CancelColor}{\color{#2}}\bcancel{#1}}

% small parallel
\makeatletter
\newcommand{\newparallel}{\mathrel{\mathpalette\new@parallel\relax}}
\newcommand{\new@parallel}[2]{%
  \begingroup
  \sbox\z@{$#1T$}% get the height of an uppercase letter
  \resizebox{!}{\ht\z@}{\raisebox{\depth}{$\m@th#1/\mkern-5mu/$}}%
  \endgroup
}
\makeatother

\renewcommand*\contentsname{Table des matières}


\begin{document}

\begin{titlepage}
  \newcommand{\HRule}{\rule{\linewidth}{0.1mm}} 
  \center % Center everything on the page
   
  %-------------------------------------------------------------------------
  %	HEADING SECTIONS
  %-------------------------------------------------------------------------
  \textsc{\Large \ncode}\\[0.5cm] % course code
  \textsc{\Large \ncourse}\\[0.5cm] % course name
  \textsc{\large \nlecturer}\\[0.5cm] % course lecturer
  
  %-------------------------------------------------------------------------
  %	TITLE SECTION
  %-------------------------------------------------------------------------
  
  \HRule \\[0.4cm]
  { \huge \bfseries  \ntitle}\\[0.1cm]
  \HRule \\[1.5cm]
  
  %\includegraphics{uclouvain.jpg} % \\[0.5cm] % 
  \vfill
\end{titlepage}

\fancyhead[R]{\thepage}
\fancyhead[L]{\rightmark}
\cfoot{}
\pagestyle{fancy}

\tableofcontents
\newpage

Le but de cet exercice est de simuler l'évolution de l'orbite de Jupiter pour les $5000$ prochaines années. Dans un premier temps nous ne tiendrons compte que de Jupiter et du Soleil (\textbf{problème à 2 corps}) et nous ajouterons par la suite la contribution de Saturne (\textbf{problème à 3 corps}).

\section{Problème à 2 corps}

Soit $m_1$ la masse du Soleil et $m_2$ la masse de Jupiter avec $m_1 \gg m_2$ et $\vec{r}_1$ la position du Soleil et $\vec{r}_2$ la position de Jupiter avec $\vec{r} = |\vec{r}_1 - \vec{r}_2|$ leur distance relative.

Prenons $\vb{q} \equiv \vb{r} = (x, y, z) \in \R^3$ comme coordonnée généralisée.

L'énergie cinétique du système est,
\begin{align*}
  T &= \sum_{i=1}^{2} T_i \\
    &= \frac{1}{2} m_1 \vd{r}_1^2 + \frac{1}{2} m_2 \vd{r}_2^2
\end{align*}

L'énergie potentielle est,
\begin{equation*}
  V = V(|\vec{r}_1 - \vec{r}_2|) = - \frac{Gm_1m_2}{|\vec{r}_1 - \vec{r}_2|}
\end{equation*}
avec $G$ la constante universelle de gravitation.

Le Lagrangien est donc, 
\begin{equation}
  \lag = \frac{1}{2} m_1 \vd{r}_1^2 + \frac{1}{2} m_2 \vd{r}_2^2 + \frac{Gm_1m_2}{|\vec{r}_1 - \vec{r}_2|}
\end{equation}

Lorsqu'on cherche une solution analytique, souvent on réexprime le Lagrangien en terme du mouvement relatif $\vb{r}$ et du mouvement du centre de masse $\vb{r}_{CM}$ et on passe en coordonnées sphériques. Ce n'est pas d'utilité dans notre cas.

L'Hamiltonien est la transformée de Legendre du Lagrangien, 
\begin{align}
  \ham 
    &= \sum_{i=1}^2 \vb{p}_i \cdot \vb{r}_i - \lag \\
    &= \frac{\vb{p}_1^2}{2m_1} + \frac{\vb{p}_2^2}{2m_2} - \frac{Gm_1m_2}{|\vb{r}_1 - \vb{r}_2|}
\end{align}
 où $\vb{p}_i$ est l'impulsion conjuguée de la i-ème masse.

 On trouve finalement les équations canoniques,
\begin{align}
  p_{1,x} &= - \pdv{\ham}{{x}_1} = - Gm_1m_2 \frac{2(x_1 - x_2)}{\left( (x_1 - x_2)^2 + (y_1 - y_2)^2 + (z_1 - z_2)^2 \right)^{3/2}} \\
  \dot{x}_1 &= \pdv{\ham}{p_{1,x}} = \frac{p_{1,x}}{m_1}
\end{align}

etc\dots

\subsection{Schémas numériques}
\subsubsection{Heun (RK2)}
Partons du schéma de Heun, 
\begin{align*}
  \begin{cases}
    \tilde{y}_{j+1} &= y_j + (\Delta t) \vb{f}_j \\
    y_{j+1} &= y_j + \frac{1}{2}(\vb{k}_1 + \vb{k}_2)
  \end{cases}
\end{align*}

Où les coefficients $\vb{k}_1$ et $\vb{k}_2$ sont donnés par, 
\begin{align*}
  \vb{k}_1 &= (\Delta t) \vb{f}_j \\
  \vb{k}_2 &= (\Delta t) \vb{f} (t_{j+1}, \tilde{y}_{j+1})
\end{align*}

Ce qui nous donne,
\begin{align}
  y(t_{j+1}) 
    &= y(t_j) + \frac{1}{2}(\Delta t) \vb{f}(t_j) + \frac{1}{2}(\Delta t) \vb{f}(t_{j+1}, \vb{y_j} + (\Delta t) \vb{f}(t_j)) \\
    &= y(t_j) + \frac{1}{2}(\Delta t) \vb{f}(t_j) + \frac{1}{2}(\Delta t) \vb{f}(t_{j+1}, \tilde{y}(t_{j+1}))
\end{align}

Effectuons un développement de Taylor de $\vb{f}(t_{j+1}, \tilde{y}(t_{j+1}))$,
\begin{align*}
  \vb{f}(t_{j+1}, \tilde{y}_{j+1}) 
    &= \vb{f}(t_j + \Delta t, \tilde{y}(t_j + \Delta t)) \\
    &= \vb{f}(t_j, \tilde{y}(t_j)) + (\Delta t) \pdv{f(t_j, y_j)}{t} + (\Delta y) \pdv{f(t_j, y_j)}{y} + \frac{1}{2} (\Delta t)^2 \pdv[2]{f(t_j, y_j)}{t} \\ 
    &+ \frac{1}{2} (\Delta y)^2 \pdv[2]{f(t_j, y_j)}{y} + \frac{1}{2} (\Delta t) (\Delta y) \pdv{f(t_j, y_j)}{t}{y} + (\Delta y) \pdv{f(t_j, y_j)}{y}
\end{align*}

\begin{definition}
  Un schéma numérique est dit \textbf{consistent} si l'itération converge vers l'expression du système original lorsque $\Delta t \rightarrow 0$
\end{definition}

On cherche à savoir si le schéma de Heun est consistent avec 
\begin{equation}
  \vd{y}(t) \equiv \vb{f}(t, y) 
\end{equation}

Pour cela, on regarde au schéma de Taylor,
\begin{align*}
  y(t_{j+1}) 
    &= y(t_j + \Delta t) \\
    &= y(t_j) + (\Delta t) y^{(1)}(t_j) + \frac{1}{2!}(\Delta t)^2 y^{(2)}(t_j) + \frac{1}{3!} (\Delta t)^3 y^{(3)}(t_j) + \dots \\
    &= y(t_j) + (\Delta t) \vb{f}(t_j, y(t_j)) + \frac{1}{2!}(\Delta t)^2 y^{(2)}(t_j) + \frac{1}{3!} (\Delta t)^3 y^{(3)}(t_j) + \dots
\end{align*}

On effectue la différence du Heun avec le schéma de Taylor et on observe que, 
\begin{equation}
  y_{\text{Heun}} - y_{\text{Taylor}} = \mathcal{O}((\Delta t)^3)
\end{equation}

Par conséquent, l'erreur commise à chaque pas de temps est de l'ordre de $(\Delta t)^3$ puisque cette différence tend vers zéro pour $(\Delta t) \rightarrow 0$, on trouve que la méthode de Heun est convergente à l'ordre $\tau = ((\Delta t)^2)$.

\begin{theorem}
  Soit une équation différentielle $y' = Ay$, où on suppose que la matrice $A \in \R^{n \times n}$ à une base de vecteurs propres et de valeurs propres $\lambda_1, \dots, \lambda_n$, la méthode de Runge Kutta a un point fixe stable (asymptotiquement stable) à l'origin lorsque appliqué à 
  \begin{equation}
    \dv{y}{t} = Ay 
  \end{equation}
  ssi le même schéma numérique à un équilibre stable (asymptotiquement stable) à l'origine lorsque apliqué à l'EDO scalaire, 
  \begin{equation}
    \dv{x}{t} = \lambda x 
  \end{equation}
  où $\lambda$ est une valeur propre de $A$.
\end{theorem}

\begin{corollary}
  Soit une EDO $\dv{y}{t} = Ay$ avec une matrice diagonalisable $A$. Soit une méthode RK donnée avec une fonction de stabilité $R$. L'origine est stable si 
  \begin{equation}
    |R(\lambda \Delta t)| \leq 1
  \end{equation}
\end{corollary}

\begin{definition} [région de stabilité]
  La région de stabilité d'un schéma numérique est la région telle que $R(\Delta t) \leq 1$  
\end{definition}

\begin{definition} [A-stabilité]
  Un schéma numérique est dit \textbf{A-stable} si les $y_N$ restent bornés dans le problème de décroissance radioactive.
\end{definition}

Afin de connaitre la \textbf{A-stabilité}, on compare la méthode de Heun avec l'équation de la décroissance radioactive. Prenons, 
\begin{equation*}
  \dv{y}{t} = - y \equiv f(t, y)
\end{equation*}

On a,
\begin{align*}
  y_{j+1}
    &= y_j - \frac{1}{2}(\Delta t) y_j - \frac{1}{2}(\Delta t) \tilde{y}_{j+1} \\
    &= y_j - \frac{1}{2}(\Delta t) y_j - \frac{1}{2}(\Delta t) (y_j - (\Delta t) y_j) \\
    &= y_j \left( 1 - \frac{1}{2}(\Delta t) - \frac{1}{2}(\Delta t) + \frac{1}{2}(\Delta t)^2 \right) \\
  y_N &= y_0 \left( 1 - \Delta t + \frac{1}{2}(\Delta t)^2 \right)^N
\end{align*}

Pour que $y_N$ soit borné il faut que, 
\begin{align*}
  \left| 1 - \Delta t + \frac{1}{2}(\Delta t)^2 \right| \leq 1 
    &\iff -1 \leq 1 - \Delta t + \frac{1}{2}(\Delta t)^2 \leq 1 \\
    &\iff -2 \leq - \Delta t + \frac{1}{2}(\Delta t)^2 \leq 0
\end{align*}

\subsubsection{RK4}
Partons du schéma de RK4, 
\begin{align*}
  y_{j+1} = y_j + \frac{1}{6}(\vb{k}_1 + 2 \vb{k}_2 + 2 \vb{k}_3 + \vb{k}_4)
\end{align*}

Où les coefficients $\vb{k}_1$, $\vb{k}_2$, $\vb{k}_3$ et $\vb{k}_4$ sont donnés par, 
\begin{align*}
  \vb{k}_1 &= (\Delta t) \vb{f}_j \\
  \vb{k}_2 &= (\Delta t) \vb{f} (t_{j} + \frac{\Delta t}{2}, \vb{y_j} + \frac{1}{2 }\vb{k}_1) \\ 
  \vb{k}_3 &= (\Delta t) \vb{f} (t_{j} + \frac{\Delta t}{2}, y_j + \frac{1}{2} \vb{k}_2) \\ 
  \vb{k}_4 &= (\Delta t) \vb{f} (t_{j}, y_{j} + \vb{k}_3) 
\end{align*}

Effectuons un développement de Taylor du schéma de RK4,
\begin{align*}
  y(t_{j+1}) 
    &= y(t_j + \Delta t) \\
    &= y(t_j) + (\Delta t) y^{(1)}(t_j) + \frac{1}{2!}(\Delta t)^2 y^{(2)}(t_j) + \frac{1}{3!} (\Delta t)^3 y^{(3)}(t_j) + \frac{1}{4!} (\Delta t)^4 y^{(4)}(t_j) + \dots \\
    &= y(t_j) + (\Delta t) \vb{f}(t_j, y(t_j)) + \frac{1}{2!}(\Delta t)^2 y^{(2)}(t_j) + \frac{1}{3!} (\Delta t)^3 y^{(3)}(t_j) + \frac{1}{4!} (\Delta t)^4 y^{(4)}(t_j) + \dots
\end{align*}

Après quelques calculs, on trouve que la différence entre le RK4 et le schéma de Taylor est de l'ordre de $((\Delta t)^5)$ ce qui implique que la méthode de RK4 est convergence à l'ordre $\tau = ((\Delta t)^4)


\end{document}