\documentclass[11pt,twoside=semi,openright,numbers=noenddot]{article}

\def\ncode {LPHYS1351}
\def\ncourse {Projet différences finies}
\def\nlecturer {M. Crucifix}
\def\nyear {2021}
\def\ntitle {Mécanique Céleste}

\usepackage{amsmath}
\usepackage{amssymb}
\usepackage{amsthm}
\usepackage[T1]{fontenc}
\usepackage[french]{babel} % traduction
%\usepackage{bookmark}
%\usepackage{booktabs} % beautiful tables
%\usepackage[thicklines]{cancel} % cancel terms in equation
%\usepackage{color} % colors in box (theorems,...)
\usepackage{framed} % box around text
\usepackage{fancyhdr} % header
%\usepackage{fancyvrb}
\usepackage{appendix}
\usepackage{mathrsfs}
\usepackage{mathtools}
\usepackage[parfill]{parskip} % avoid bad alignment in paragraphs
\usepackage{pgfplots} % plot in tikz
\usepackage{braket} % braket and sets
\usepackage{derivative} % derivatives
%\usepackage{setspace}
\usepackage{siunitx}
\usepackage{silence} % silence useless warnings
%\usepackage{systeme} % system of equations
\usepackage{tikz}
\usepackage{tkz-euclide}
\usepackage{tikz-3dplot}
\usepackage{units} % Non-stacked fractions and better unit spacing
\usepackage{xspace} % prints a trailing space in a smart way.

\usepackage[font={small,it}]{caption}
\captionsetup[figure]{font=scriptsize}%labelformat=empty}

\usetikzlibrary{calc,patterns,angles,quotes}
\usetikzlibrary{hobby} % random curves with tikz

\tikzset{circ/.style = {fill, circle, inner sep = 0, minimum size = 3}}
\tikzset{scirc/.style = {fill, circle, inner sep = 0, minimum size = 1.5}}
\tikzset{mstate/.style={circle, draw, blue, text=black, minimum width=0.7cm}}

% increase vertical space for aligned equations
\setlength{\jot}{7pt}

%----------------------
%	COMMANDS
%----------------------

% commands shortcuts
\newcommand{\mb}{\mathbb}
\newcommand{\R}{\mb{R}}
\newcommand{\Z}{\mb{Z}}
\newcommand{\N}{\mb{N}}
\newcommand{\C}{\mb{C}}
\newcommand{\A}{\mb{A}} % hypersphere area
\newcommand{\V}{\mb{V}} % hypersphere volume
\newcommand{\dS}{\cdot d\vb{S}}
\newcommand{\lag}{\mathcal{L}}
\newcommand{\ham}{\mathcal{H}}
\newcommand{\Mod}[1]{\ \mathrm{mod}\ #1}

\renewcommand{\vec}[1]{\mathbf{#1}} % bold vectors
\newcommand{\vd}[1]{\dot{\vec{#1}}}
\newcommand{\vdd}[1]{\ddot{\vec{#1}}}

% norm
\newcommand{\bignorm}[1]{\left\lVert#1\right\rVert}

% smaller overline (line above variable)
\newcommand{\overbar}[1]{\mkern 1.5mu\overline{\mkern-1.5mu#1\mkern-1.5mu}\mkern 1.5mu}

% prints an asterisk that takes up no horizontal space.
% useful in tabular environments.
\newcommand{\hangstar}{\makebox[0pt][l]{*}}

% Prints argument within hanging parentheses (i.e., parentheses that take
% up no horizontal space). Useful in tabular environments.
\newcommand{\hangp}[1]{\makebox[0pt][r]{(}#1\makebox[0pt][l]{)}}

% cancel terms with color
% \newcommand{\ccancel}[2]{\renewcommand{\CancelColor}{\color{#2}}\bcancel{#1}}

% small parallel
\makeatletter
\newcommand{\newparallel}{\mathrel{\mathpalette\new@parallel\relax}}
\newcommand{\new@parallel}[2]{%
  \begingroup
  \sbox\z@{$#1T$}% get the height of an uppercase letter
  \resizebox{!}{\ht\z@}{\raisebox{\depth}{$\m@th#1/\mkern-5mu/$}}%
  \endgroup
}
\makeatother


\begin{document}

\begin{titlepage}
  \newcommand{\HRule}{\rule{\linewidth}{0.1mm}} 
  \center % Center everything on the page
   
  %-------------------------------------------------------------------------
  %	HEADING SECTIONS
  %-------------------------------------------------------------------------
  \textsc{\Large \ncode}\\[0.5cm] % course code
  \textsc{\Large \ncourse}\\[0.5cm] % course name
  \textsc{\large \nlecturer}\\[0.5cm] % course lecturer
  
  %-------------------------------------------------------------------------
  %	TITLE SECTION
  %-------------------------------------------------------------------------
  
  \HRule \\[0.4cm]
  { \huge \bfseries  \ntitle}\\[0.1cm]
  \HRule \\[1.5cm]
  
  %\includegraphics{uclouvain.jpg} % \\[0.5cm] % 
  \vfill
\end{titlepage}

\fancyhead[R]{\thepage}
\fancyhead[L]{\rightmark}
\cfoot{}
\pagestyle{fancy}

\tableofcontents
\newpage

Le but de cet exercice est de simuler l'évolution de l'orbite de Jupiter pour les $5000$ prochaines années. Dans un premier temps nous ne tiendrons compte que de Jupiter et du Soleil (\textbf{problème à 2 corps}) et nous ajouterons par la suite la contribution de Saturne (\textbf{problème à 3 corps}).

\section{Problème à 2 corps}

Soit $m_1$ la masse du Soleil et $m_2$ la masse de Jupiter avec $m_1 \gg m_2$ et $\vec{r}_1$ la position du Soleil et $\vec{r}_2$ la position de Jupiter avec $\vec{r} = |\vec{r}_1 - \vec{r}_2|$ leur distance relative.

Prenons $\vb{q} = \vb{r} \in \R^3$ comme coordonnée généralisée.

L'énergie cinétique du système est,
\begin{align*}
  T &= \sum_{i=1}^{2} T_i \\
    &= \frac{1}{2} m_1 \vd{r}_1^2 + \frac{1}{2} m_2 \vd{r}_2^2
\end{align*}

L'énergie potentielle est,
\begin{equation*}
  V = V(|\vec{r}_1 - \vec{r}_2|) = - \frac{Gm_1m_2}{|\vec{r}_1 - \vec{r}_2|}
\end{equation*}
avec $G$ la constante universelle de gravitation.

Le Lagrangien est donc, 
\begin{equation}
  \lag = \frac{1}{2} m_1 \vd{r}_1^2 + \frac{1}{2} m_2 \vd{r}_2^2 + \frac{Gm_1m_2}{|\vec{r}_1 - \vec{r}_2|}
\end{equation}

Lorsqu'on cherche une solution analytique, souvent on réexprime le Lagrangien en terme du mouvement relatif $\vb{r}$ et du mouvement du centre de masse $\vb{r}_{CM}$ et on passe en coordonnées sphériques. Ce n'est pas d'utilité dans notre cas.

L'Hamiltonien est la transformée de Legendre, 
\begin{align}
  \ham 
    &= \sum_{i=1}^2 \vb{p}_i \cdot \vb{r}_i - \lag \\
    &= \frac{\vb{p}_1^2}{2m_1} + \frac{\vb{p}_2^2}{2m_1} - \frac{Gm_1m_2}{|\vec{r}_1 - \vec{r}_2|}
\end{align}
 où $\vb{p}_i$ est l'impulsion conjuguée de la i-ème masse.

 On trouve finalement les équations canoniques,
\begin{align}
  \vd{p}_i &= - \pdv{\ham}{\vb{q}_i} \\
  \vd{q}_i &= \pdv{\ham}{\vb{p}_i} = \frac{\vb{p}_i}{m_i}
\end{align}

\subsection{Schémas numériques}
Posons, 
\begin{equation}
  \vb{y} = 
  \begin{pmatrix}
    p_1 \\ p_2 \\ p_3 \\ q_1 \\ q_2 \\ q_3
  \end{pmatrix}
\end{equation}

\subsubsection{Heun (RK2)}
Partons du schéma de Heun, 
\begin{align*}
  \begin{cases}
    \tilde{\vb{y}}_{j+1} &= \vb{y}_j + (\Delta t) \vb{f}_j \\
    \vb{y}_{j+1} &= \vb{y}_j + \frac{1}{2}(\vb{k}_1 + \vb{k}_2)
  \end{cases}
\end{align*}

Où les coefficients $\vb{k}_1$ et $\vb{k}_2$ sont donnés par, 
\begin{align*}
  \vb{k}_1 &= (\Delta t) \vb{f}_j \\
  \vb{k}_2 &= (\Delta t) \vb{f} (t_{j+1}, \vb{y_j} + \vb{k}_1)
\end{align*}

Ce qui nous donne,
\begin{align}
  \vb{y}(t_{j+1}) 
    &= \vb{y}(t_j) + \frac{1}{2}(\Delta t) \vb{f}(t_j) + \frac{1}{2}(\Delta t) \vb{f}(t_{j+1}, \vb{y_j} + (\Delta t) \vb{f}(t_j)) \\
    &= \vb{y}(t_j) + \frac{1}{2}(\Delta t) \vb{f}(t_j) + \frac{1}{2}(\Delta t) \vb{f}(t_{j+1}, \tilde{\vb{y}}(t_{j+1}))
\end{align}

On cherche à savoir s'il est consistent avec 
\begin{equation}
  \vd{y}(t) \equiv \vb{f}(t, \vb{y}) 
\end{equation}

Effectuons un développement de Taylor,
\begin{align*}
  \vb{f}(t_{j+1}, \vb{y}(t_{j+1})) 
    &= \vb{f}(t_j + \Delta t, \vb{y}(t_j + \Delta t)) \\
    &= \vb{f}(t_j, \vb{y}(t_j)) + (\Delta t) \pdv{f(t_j, \vb{y}_j)}{t} + (\Delta y) \pdv{f(t_j, \vb{y}_j)}{y} + \frac{1}{2} (\Delta t)^2 \pdv{f(t_j, \vb{y}_j)}{t^2} + \frac{1}{2} (\Delta y)^2 \pdv{f(t_j, \vb{y}_j)}{y^2} + \frac{1}{2} (\Delta t) (\Delta y) \pdv{f(t_j, \vb{y}_j)}{t, y} + (\Delta y) \pdv{f(t_j, \vb{y}_j)}{y}
\end{align*}

On regarde au schéma de Taylor,
\begin{align*}
  \vb{y}(t_{j+1}) 
    &= \vb{y}(t_j + \Delta t) \\
    &= \vb{y}(t_j) + (\Delta t) \vb{y}^{(1)}(t_j) + \frac{1}{2!}(\Delta t)^2 \vb{y}^{(2)}(t_j) + \frac{1}{3!} (\Delta t)^3 y^{(3)}(t_j) + \dots \\
    &= \vb{y}(t_j) + (\Delta t) \vb{f}(t_j, \vb{y}(t_j)) + \frac{1}{2!}(\Delta t)^2 \vb{y}^{(2)}(t_j) + \frac{1}{3!} (\Delta t)^3 y^{(3)}(t_j) + \dots
\end{align*}

On compare Heun avec ce développement de Taylor, 

\subsubsection{RK4}
Partons du schéma de RK4, 
\begin{align*}
  \vb{y}_{j+1} = \vb{y}_j + \frac{1}{6}(\vb{k}_1 + 2 \vb{k}_2 + 2 \vb{k}_3 + \vb{k}_4)
\end{align*}

Où les coefficients $\vb{k}_1$, $\vb{k}_2$, $\vb{k}_3$ et $\vb{k}_4$ sont donnés par, 
\begin{align*}
  \vb{k}_1 &= (\Delta t) \vb{f}_j \\
  \vb{k}_2 &= (\Delta t) \vb{f} (t_{j} + \frac{\Delta t}{2}, \vb{y_j} + \frac{1}{2 }\vb{k}_1) \\ 
  \vb{k}_3 &= (\Delta t) \vb{f} (t_{j} + \frac{\Delta t}{2}, y_j + \frac{1}{2} \vb{k}_2) \\ 
  \vb{k}_4 &= (\Delta t) \vb{f} (t_{j}, y_{j} + \vb{k}_3) 
\end{align*}

Effectuons un développement de Taylor du schéma de RK4,
\begin{align*}
  \vb{y}(t_{j+1}) 
    &= \vb{y}(t_j + \Delta t) \\
    &= \vb{y}(t_j) + (\Delta t) \vb{y}^{(1)}(t_j) + \frac{1}{2!}(\Delta t)^2 \vb{y}^{(2)}(t_j) + \frac{1}{3!} (\Delta t)^3 y^{(3)}(t_j) + \frac{1}{4!} (\Delta t)^4 y^{(4)}(t_j) + \dots \\
    &= \vb{y}(t_j) + (\Delta t) \vb{f}(t_j, \vb{y}(t_j)) + \frac{1}{2!}(\Delta t)^2 \vb{y}^{(2)}(t_j) + \frac{1}{3!} (\Delta t)^3 y^{(3)}(t_j) + \frac{1}{4!} (\Delta t)^4 y^{(4)}(t_j) + \dots
\end{align*}


\end{document}